\documentclass[10pt,preprint]{aastex}
\usepackage[top=1in,right=1in,left=1in]{geometry}
\usepackage{verbatim}
\usepackage{hyperref}

\usepackage{savesym}
\savesymbol{tablenum}
\usepackage{siunitx}
\restoresymbol{SIX}{tablenum}

\usepackage[svgnames]{xcolor}
\definecolor{zz}{RGB}{0,0,255}
\hypersetup{
    colorlinks=true,
    linkcolor=zz,
    filecolor=magenta,      
    urlcolor=zz,
}

\newcommand{\jj}[1]{\textcolor{red}{#1}}

\urlstyle{same}

\newcommand*{\xdash}[1][3em]{\rule[0.5ex]{#1}{0.7pt}}

\pagestyle{plain} %to adjust page numbering

\begin{document}
\begin{center}
{\bf{\large Jeff Jennings}}
\end{center}
Eberly Research Fellow \hfill
\hfill jeffj@psu.edu\\
Pennsylvania State University \hfill
\hfill \href{https://ui.adsabs.harvard.edu/search/q=orcid\%3A0000-0002-7032-2350&sort=date+desc}{ADS}, \href{http://bit.ly/jennings_googlescholar}{Google Scholar}, \href{http://github.com/jeffjennings}{GitHub}

\noindent {\bf Education and research experience} \xdash[69ex] \\ %\xdash[97ex] \\
08.2022 - Present \-\hspace{1.3cm} Eberly Research Fellow, Astronomy, Pennsylvania State University \\
\-\hspace{4.51cm}Image synthesis for radio interferometry at the petabyte scale

\noindent 2018 - 2022 \-\hspace{2.2cm} Ph.D., Astronomy, University of Cambridge \\
(4 yr) \-\hspace{3.5cm}Thesis: Characterizing sub-mm observations of protoplanetary disks at \\
\-\hspace{4.4cm} super-resolution scales -- Advisor Cathie Clarke

\noindent 2017 \-\hspace{3.25cm} Visiting student researcher, Astronomy, University of Washington\\
(9 month) \-\hspace{2.85cm}Statistical analysis of exoplanet transit timing variations -- Advisor Eric Agol

\noindent 2016 - 2017 \-\hspace{2.2cm} M.Sc., Astrophysics, Ludwig-Maximilians-Universit{\"a}t M{\"u}nchen (LMU Munich)\\
(1 yr 5 month) \-\hspace{2.1cm}Thesis: Energetic regimes of photoevaporative disc dispersal \& their influence \\
\-\hspace{4.55cm}on gas giant migration -- Advisor Barbara Ercolano

\noindent 2015 - 2016; 2017 - 2018 \-\hspace{.22cm} Research assistant, Optics, National Institute of Standards and Technology\\
% Optical Frequency Measurements Group
(1 yr 9 month) \-\hspace{2.2cm}Experimental development of frequency combs and interferometers for extreme \\ \-\hspace{4.55cm}precision radial velocity surveys -- Advisor Scott Diddams 

\noindent 2013 - 2015 \-\hspace{2.2cm} Post-baccalaureate, Physics, University of Colorado at Boulder \\
(2 yr 5 month) \-\hspace{2.1cm}Observational luminosity class diagnostics -- Advisor Emily Levesque

\noindent 2008 - 2011 \-\hspace{2.2cm} B.A., University of Colorado at Boulder \\
(3 yr 5 month) \-\hspace{2.1cm}Ecology \& Evolutionary Biology (Major 1), Environmental Studies (Major 2) 


% \noindent {\bf Research foci} \xdash[93.0ex] \\
% \underline{Thematic} \\
% \indent Statistical characterization of exoplanetary systems embedded in protoplanetary discs \\
% \underline{Technical} \\ 
% \indent Image synthesis for interferometric observations in the mm and optical

% \noindent {\bf Skills} \xdash[102.3ex] \\
% \noindent {\bf Software engineering:} Python, C++, SQL, HTML, bash, git, Docker, Linux/Mac/Windows, hardware acceleration, high performance computing, open-source software development and maintenance, documentation, user support and custom solutions

% \noindent {\bf Data science, machine learning:} Fourier analysis, time series analysis, noise characterization, optimization, parametric and nonparametric regression, statistics and Bayesian inference, classification, big data processing and analytics, concise visualizations

% \noindent {\bf Project management:} Guiding a project via high-level strategy and consideration of low-level nuances; resolving difficult or novel challenges by approaching them from first principles; designing and implementing simple, automated, end-to-end solutions to complex problems; balancing multiple projects concurrently; leading 3 – 20 person collaborative efforts through multi-year goals; coordinating with other teams to solve interdisciplinary problems; clearly communicating concepts at multiple levels of complexity, in familiar vocabulary; mentoring and training researchers; developing course materials and teaching

% \noindent {\bf Projects} \xdash[98.7ex] \\
% - Designing {\tt morticia}, a pipeline to autonomously query, acquire, reduce, model, and analyze $\approx 1$~PB of archival ALMA \\ \indent continuum and spectral line observations of protoplanetary disks \\
% - Leading the image analysis team for the ALMA large program ARKS (\lq{}ALMA survey to Resolve exoKuiper \\ \indent belt Substructures\rq{}) of 20 debris disk sources \\
% - Co-developed two open-source software packages for image construction from Fourier data, prioritizing \\ \indent CI/CD principles (tests, documentation, user support for modular code usage and one-line execution) \\
% - Led experiments, data processing, analysis, biweekly meetings for two-year collaboration of physicists, \\ \indent astronomers and engineers to characterize two interferometers subsequently deployed to telescopes

% - Co-developer and maintainer of the open-source software package {\tt frank} for 1D interferometric imaging in sub-mm astronomy, used broadly in the protoplanetary disk and debris disk fields (40+ peer-reviewed citations since 2020)
% - Co-developer and maintainer of the open-source software package {\tt MPoL} for 2D interferometric imaging (25+ stars on GitHub)
% - As a team lead in the ALMA large program ARKS, developed the open-source pipeline {\tt arksia} to implement interoperability and reproducibility of interferometric imaging
% - Developed and tested joint hardware-software calibration systems for optical/IR spectroscopy
% - Designing a pipeline in SQL and Python to autonomously query, acquire, process and reduce ~1 PB of astronomical observations in the ALMA archive using high-performance computing

\noindent {\bf Published works} \xdash[89.5ex] \\
\centerline{{\bf 8 first/second author, 11 third/later author, h-index 11 (first/second author h-index 7)}}

\noindent \underline{{\bf Refereed: first and second author}} \\
\noindent Super-resolution trends in the ALMA Taurus survey: structured inner discs and compact discs \\
\indent {\bf J. Jennings}, M. Tazzari, C. J. Clarke, R. Booth, \& G. P. Rosotti 2022 MNRAS 

\noindent A super-resolution analysis of the DSHARP survey: substructure is common in the inner 30 au \\
\indent {\bf J. Jennings}, R. Booth, M. Tazzari, C. J. Clarke, \& G. P. Rosotti 2022 MNRAS

\noindent Frankenstein: protoplanetary disc brightness profile reconstruction at sub-beam resolution with a rapid Gaussian process \\
\indent {\bf J. Jennings}, R. Booth, M. Tazzari, G. P. Rosotti, \& C. J. Clarke 2020 MNRAS

\noindent Frequency stability of the mode spectrum of broad bandwidth Fabry-P{\'e}rot interferometers \\
\indent {\bf J. Jennings}, R. Terrien, C. Fredrick, M. Grisham, M. Notcutt, S. Halverson, S. Mahadevan, \\ 
\indent \& S. A. Diddams 2020 OSA Continuum 

\noindent The comparative effect of FUV, EUV and X-ray disc photoevaporation on gas giant separations \\
\indent {\bf J. Jennings}, B. Ercolano \& G. Rosotti 2018 MNRAS 

\noindent X-ray photoevaporation's limited success in the formation of planetesimals by the streaming instability\\
\indent B. Ercolano, {\bf J. Jennings}, G. Rosotti, \& T. Birnstiel 2017 MNRAS

\noindent Frequency stability characterization of a broadband fiber Fabry-P{\'e}rot interferometer \\
\indent {\bf J. Jennings}, S. Halverson, R. Terrien, S. Mahadevan, G. Ycas, \& S. A. Diddams 2017 Optics Express

\noindent H$\alpha$ as a luminosity class diagnostic for K- and M-type stars \\
\indent {\bf J. Jennings} \& E. M. Levesque 2016 ApJ 

\noindent \underline{{\bf Refereed: third and later author}} \\
\noindent High resolution ALMA observations of compact discs in the wide binary system of Sz~65 and Sz~66\\
\indent J. Miley et al. A\&A (submitted)

\noindent Regularized maximum likelihood image synthesis and validation for ALMA continuum
observations of protoplanetary disks \\
\indent B. Zawadzki et al. 2023 PASP

\noindent Deprojecting and constraining the vertical thickness of exoKuiper belts \\
\indent J. Terrill et al. 2023 MNRAS

\noindent Distribution of solids in the rings of the HD 163296 disk: a multi-wavelength study \\
\indent G. Guidi et al. 2022 A\&A 

\noindent Unveiling the outer dust disc of TW Hya with deep ALMA observations \\
\indent J. Ilee et al. 2022 MNRASL

\noindent Broadband stability of the Habitable Zone Planet Finder Fabry-P{\'e}rot etalon calibration system: evidence for chromatic variation \\
\indent R. Terrien et al. 2021 AJ 

\noindent Dust ring morphology in protoplanetary disks from ALMA dual-wavelength observations \\
\indent F. Long et al. 2020 ApJ 

\noindent Modelling thermochemical processes in protoplanetary
disks I: numerical methods \\
\indent T. Grassi et al. 2020 MNRAS

\noindent Evidence for He I 10830 \si{\angstrom} absorption during the transit of a warm Neptune around the M-dwarf GJ 3470 with the Habitable-zone Planet Finder \\
\indent J. P. Ninan et al. 2020 ApJ

\noindent A sub-Neptune sized planet transiting the M2.5-dwarf G 9-40: validation with the Habitable-zone Planet Finder \\
\indent G. Stefansson et al. 2020 AJ

\noindent Stellar spectroscopy in the near-infrared with a laser frequency comb \\
\indent A. J. Metcalf et al. 2019 Optica 

\noindent \underline{{\bf Conference proceedings}} \\
\noindent Infrared astronomical spectroscopy for radial velocity
measurements with 10 cm s$^{-1}$ precision \\
\indent A. J. Metcalf et al. 2018 CLEO

\noindent Measuring the thermal sensitivity of a fiber Fabry-P{\'e}rot interferometer \\
\indent {\bf J. Jennings}, S. Halverson, S. A. Diddams, R. Terrien, G. Ycas, \& S. Mahadevan 2016 Proc.~of SPIE

\noindent {\bf Software} \xdash[98.5ex] \\
\noindent Frankenstein (\href{https://discsim.github.io/frank/}{\tt frank}) -- Python package for super-resolution, 1D imaging of sub-mm interferometric data. \\ \-\hspace{3.65cm}Implements a non-parametric, empirical Bayes approach with a Gaussian process

\noindent Million Points of Light (\href{https://mpol-dev.github.io/MPoL/index.html}{\tt MPoL}) -- Python package for super-resolution, 2D imaging of interferometric data. \\ \-\hspace{3.65cm}Implements machine learning techniques for regularized maximum likelihood imaging

\noindent \href{https://github.com/jeffjennings/arksia/}{\tt arksia} -- Python package for bulk imaging and characterization of the ALMA large program ARKS datasets. \\ \-\hspace{3.65cm}Uses parallelization and custom routines to scale imaging and analysis for 20 sources
\noindent {\tt morticia} -- Pipeline (prototype stage) to self-consistently image and analyze $\approx 1$~PB of archival ALMA \\ \-\hspace{3.65cm}continuum and spectral line observations of protoplanetary disks %\\
% \noindent \href{https://github.com/jeffjennings/morticia/}{\tt morticia} -- 
% Python package for imaging and analysis of hundreds of datasets from the ALMA archive. \\ \-\hspace{3.65cm}\jj{Uses hardware acceleration, HPC to scale an end-to-end pipeline for $\approx 1$~PB of data}

\noindent {\bf Peer review} \xdash[95ex] \\
\noindent Referee of 8 published papers: \\
\indent 4 in MNRAS, MNRASL on disc photoevaporation, disc substructure \\
\indent 2 in ApJL, ApJS on disc substructure, interferometric imaging \\
\indent 1 in A\&A on disc substructure \\
\indent 1 in JOSS (Journal of Open Source Software) on interferometric imaging \\
\noindent NASA peer review: \\
\indent 2 panels in 2023, Chair of 1 panel

\noindent {\bf Recent talks} \xdash[94ex] \\
\noindent Regularized maximum likelihood imaging for sub-mm astronomy \\
\indent 09.2023 -- Lunch talk -- Columbia University \\
\noindent ALMA imaging beyond CLEAN: techniques and applications \\
\indent 08.2023 -- Star and planet formation seminar -- ESO Headquarters \\
\noindent The scientific utility of software development: a case study in radio interferometry \\
\indent 03.2023 -- Astronomy \& astrophysics colloquium -- Pennsylvania State University

\noindent {\bf Teaching and outreach} \xdash[83ex] \\
\noindent Cambridge Ph.D. student hack week -- Organizer (11.2019), 15 attendees \\
%\noindent Conference breakout session co-host\\
%\indent Collaboration between the star formation and planet-forming disc communities\\
%\indent Rocky Worlds: from the Solar System to Exoplanets\\
%\indent 01.2020 -- Cambridge, UK 
\noindent Cambridge \lq{}Ask an Astronomer\rq{} outreach program (Q\&A at public open nights) -- Organizer (2019/2020) \\
\noindent Cambridge University Astronomy %\jj{\href{https://www.youtube.com/CambridgeUniversityAstronomy}{YouTube channel}} 
YouTube channel (outreach platform) -- Co-founder (03.2020) \\
% \noindent IoA Forum on Racism and Racial Equality -- Member (06.2020 -- 09.2021) \\
% \noindent IoA Equality \& Diversity Committee -- Member (03.2020 -- 09.2021) \\
% \noindent IoA Graduate Student Forum -- Representative (10.2018 -- 09.2021) \\
\noindent Undergraduate supervision -- Tutorial sessions for introductory physics (10.2018 -- 06.2019)
% \indent Tutorial sessions for four students in introductory physics -- Clare College, University of Cambridge

% \noindent {\bf Observing proposals} \xdash[85ex] \\
% An unbiased census of disc structures and dust evolution in Solar System progenitors \\
% \indent P.I. M. Tazzari, ALMA Cycle 7 (2019, unsuccessful, Grade \jj{xx})

% \noindent {\bf Funding proposals} \xdash[87ex] \\
% \noindent \underline{Single author} \\
% \noindent Cambridge Ph.D. student hack week -- Departmental support -- 2019, GBP 600 awarded \\

% \noindent {\bf Instrument deployment} \xdash[81ex] \\
% Near-IR interferometer, a wavelength calibrator for the Habitable-zone Planet Finder spectrograph \\
% \indent Hobby-Eberly Telescope (10 m), McDonald Observatory, 2018

% \noindent {\bf Presentations} \xdash[92.3ex] \\
% \noindent \underline{{\bf Conference talks}} \\
% \noindent What can super-resolution imaging tell us about substructure origins? (invited student talk)\\
% \indent 10.2021 -- Structure formation in planet-forming disks -- Garching, Germany\\
% \noindent Understanding super-resolution dust substructure trends in the DSHARP survey \\
% \indent 12.2020 -- Dustbusters collaboration mid-term meeting -- Virtual
% \noindent New dust substructure trends in DSHARP: a super-resolution analysis with Frankenstein \\
% \indent 12.2020 -- Five years after HL Tau: a new era in planet formation -- Virtual\\
% \noindent Frankenstein: protoplanetary disc brightness profile recovery (poster award talk) \\
% \indent 07.2019 -- Great barriers in planet formation -- Palm Cove, Australia %\\
% \noindent Assessing a bias in planet masses recovered with transit timing variations\\
% \indent 08.2017 -- Exoclipse 2017: exploring new worlds in the shade -- Boise, ID, USA \\
%\indent Awarded travel grant
% \noindent Photoevaporative disc dispersal as a weak planetesimal formation mechanism\\
% \indent 06.2017 -- The formation and evolution of planets and their discs -- Garching, Germany \\
% \noindent The effects of photoevaporation on gas giant migration in protoplanetary discs \\
% \indent 03.2017 -- Formation and dynamical evolution of exoplanets -- Aspen, CO, USA
%\indent Awarded travel grant

% \noindent \underline{{\bf Department talks}} \\
% \noindent ALMA imaging beyond CLEAN: techniques and applications \\
% \indent 08.2023 -- Star and planet formation seminar -- ESO Headquarters \\
% \noindent The scientific utility of software development: a case study in radio interferometry \\
% \indent 03.2023 -- Astronomy \& astrophysics colloquium -- Pennsylvania State University \\
% \noindent Super-resolution views of continuum structure in protoplanetary disks \\
% \indent 12.2021 -- Virginia Initiative On Cosmic Origins (VICO) workshop -- University of Virginia \\
% \noindent Investigating planet formation with super-resolution imaging \\
% \indent 10.2021 -- Center for Exoplanets \& Habitable Worlds seminar -- Pennsylvania State University \\
% \noindent Image synthesis in the mm: first results with the sub-beam model Frankenstein \\
% \indent 04.2020 -- Theoretical astrophysics group lunchtime talk -- University of Leicester \\
% \noindent Extracting higher resolution information from sub-mm observations \\
% \indent 12.2019 -- Protoplanetary disks group meeting -- Center for Astrophysics $|$ Harvard \& Smithsonian \\
% \noindent Recovering sky brightness from radio interferometry: a Gaussian process approach
% \indent 11.2019 -- Institute of Astronomy seminar -- University of Cambridge \\
% \noindent Protoplanetary disc brightness profile reconstruction at super-CLEAN resolution in $<$1 minute \\
% \indent 10.2019 -- Star and planet formation seminar -- ESO Headquarters 
%\noindent Constraining photoevaporative effects on protoplanetary disc evolution \\
%\indent 11.2016 -- Protoplanetary disc research consortium annual meeting -- Garching, Germany
% \noindent \underline{{\bf Conference poster awards}} \\
% \noindent 07.2019 -- Great barriers in planet formation -- Palm Cove, Australia \\
% \indent Best poster award -- 1st place \\
% \noindent 11.2016 -- Exoplanets: bridging the gap between theory and observations -- Bonn, Germany \\
% \indent Best poster award -- 2nd place

% \noindent \underline{Conference posters} (presented at the following) \\
% \noindent 12.2020 -- Five years after HL Tau: a new era in planet formation -- Virtual \\
% \noindent 01.2020 -- Rocky worlds: from the solar system to exoplanets -- Cambridge, UK \\
%\noindent Frankenstein: protoplanetary disc brightness profile reconstruction in seconds with a Gaussian process\\
% \noindent 07.2019 -- Great barriers in planet formation -- Palm Cove, Australia \\
% \indent \indent \indent Best poster award -- 1st place \\
%\noindent When, how and why the TTV mass--eccentricity degeneracy can bias recovered planet masses\\
% \noindent 05.2019 -- New horizons in planetary systems -- Victoria, BC, Canada \\
%\indent Awarded travel grant
%\noindent When, how and why the TTV mass--eccentricity degeneracy can bias recovered planet masses\\
% \iffalse
% \noindent 04.2019 -- UK exoplanet community meeting 2019 -- London, UK \\
%\noindent When and how the TTV mass--eccentricity degeneracy can bias recovered planet masses\\
% \noindent 12.2018 -- PLATO theory workshop 2018 -- Cambridge, UK \\
%\noindent The difficulty of forming Earth analogs\\
% \noindent 11.2017 -- Habitable worlds 2017 -- Laramie, WY, USA \\
%\indent Awarded travel grant
%\noindent Photoevaporation, gas giant migration and initial planetary system architectures\\
% \noindent 05.2017 -- The disc migration issue -- Cambridge, UK \\
%\indent Awarded travel grant
%\noindent The comparative effect of FUV, EUV and X-ray photoevaporation on gas giant parking radii \\
% \noindent 11.2016 -- Exoplanets: bridging the gap between theory and observations -- Bonn, Germany \\
% \indent \indent \indent Best poster award -- 2nd place
%\indent Awarded travel grant
%\noindent Frequency stability characterization of a broadband fiber Fabry-P{\'e}rot interferometer \\
% \noindent 06.2016 -- SPIE astronomical telescopes \& instrumentation -- Edinburgh, UK

% \noindent {\bf Summer schools} \xdash[89.5ex] \\
% \noindent 08.2020 -- Protostellar discs and planet formation -- Cefalu, Italy \\
% \noindent 06.2020 -- NRAO 17$^{\rm th}$ synthesis imaging workshop -- Socorro, NM, USA (remote) \\ 
% \noindent 05.2019 -- Demographics of exoplanetary systems -- Vietri sul Mare, Italy 
%\indent Awarded travel grant
% \noindent 08.2017 -- Astrophysical plasmas: from planets to galaxies -- Copenhagen, Denmark%, 2.5 ECTS
%\indent Awarded travel grant

% \noindent \jj{Volunteer work} \xdash[72.2ex] \\
% \noindent 07.2014 - 08.2014 \quad	International Anti-Poaching Foundation, Victoria Falls, Zimbabwe
% \noindent 06.2014 - 07.2014 \quad Elephant-Human Relations Aid, Swakopmund, Namibia

\end{document}
